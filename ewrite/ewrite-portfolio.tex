\documentclass[11pt,english]{article}

\usepackage[english]{babel}
\usepackage{enumitem}
\usepackage{fancyhdr}
\usepackage[top=2.3cm, bottom=2.3cm, left=1.6cm, right=1.6cm]{geometry}
\usepackage{graphicx}
\usepackage{lastpage}
\usepackage{standalone}

\setlength{\parindent}{0pt}        % indentation on new paragraph
\setlength{\parskip}{0pt}          % vertical spacing on new paragraph
\setlength{\lineskip}{1pt}         % vertical spacing between lines
\setlength{\columnsep}{1cm}        % spacing between columns
\setlength{\belowcaptionskip}{0pt} % spacing below captions
\setlength{\abovecaptionskip}{5pt} % spacong above captions

\pagestyle{fancy}

\graphicspath{{assets/}}

\lfoot{Patrick Spek}
\cfoot{\today}
\rfoot{\thepage/\pageref{LastPage}}

\newcommand*{\Frontpage}{\begingroup
	\hbox{%
		\hspace*{0.2\textwidth}
		\rule{1pt}{\textheight}
		\hspace*{0.05\textwidth}
		\parbox[b]{0.75\textwidth}{%
			{\noindent\Huge\bfseries Portfolio}\\[2\baselineskip] % Title
			{\large \textit{English writing}}\\[4\baselineskip] % Tagline or further description
			{\Large \textsc{\\
				Patrick Spek \\
			}}

			\vspace{0.5\textheight} % Whitespace between the title block and the publisher
		}
	}
\endgroup}

\renewcommand{\footrulewidth}{0.4pt}

% additional markup for included documents
\usepackage{hyperref}
\providecommand{\tightlist}{\setlength{\itemsep}{0pt}\setlength{\parskip}{0pt}}

\begin{document}
	% frontpage
	\thispagestyle{empty}
	\Frontpage{}

	% foreword
	\newpage
	\section{Foreword}
	This document is the portfolio for my English writing course at Avans. It
	is intended to help me learn about writing in a professional manner which I
	can use in my later career. At the start of our course I was asked to write
	down a number of goals I wanted to improve on during this course.

	\paragraph{}
	One of these goals is to improve my spelling. While there are dictionaries
	available to correct my mistakes, it would save me time and effort to
	simply do it right in a single try. This is something I wish to improve on
	during the course.

	\paragraph{}
	I tend to write a lengthy piece of text once I get started on it. However,
	this can be a chore to read for the other party, and even make the intended
	message harder to read. In turn, I want to improve on this by making
	concise messages that clearly get the intended message across, while still
	leaving enough details for the recipient to understand the reasoning behind
	the message.

	\paragraph{}
	Another of my learning goals for this class is to make less errors with
	regards to the word ordering. In formal English texts (and not-so-formal
	texts as well, for that matter), the order in which words appear is
	important. I often make mistakes in this area, for instance by putting the
	time somewhere in the middle of the sentence instead of at the end.

	\paragraph{}
	In the rest of this document you will find my results of the assignments I
	was given for this course. This includes three different documents. The
	first is an email with a complaint. The second assignment is an executive
	summary on a document made for an assignment in the first year of
	university. Lastly, there is a tutorial document. This is a rather tough
	piece of text intended for those with a little more experience in the IT
	field.

	\paragraph{}
	In addition to these finalized assignments, this portfolio contains their
	earlier versions. These were handed in to be rated by our instructor so
	that I can improve on it.

	% table of contents
	\newpage
	\tableofcontents

	% main body
	\newpage
	\section{The Email}
	\subsection{Old revision}
	\documentclass[11pt,english]{article}

\usepackage[english]{babel}

\begin{document}
  \textbf{Subject: Laptop issues}

  \paragraph{}
  Dear Sir or Madam,
\\ \\
  I am writing to inform you about issues we have with the laptops we recently
  ordered from your company, Lenovo.
\\ \\ 
  I am afraid that the batch of 600 IdeaPads we recently ordered and received
  are having technical issues which are impairing our current workflow. The
  laptops seem to have issues connecting to our WiFi network, even after we
  applied all firmware updates.
\\ \\
  Another issue we have ran into is battery failure. Even though ACPI reports
  the batteries to have well over one hour of battery life left, the machine
  shuts off and will only turn on when the laptop is connected to a power source
  or a fully charged battery is inserted.
\\ \\
  I have also received a number of complaints from my employees about the
  keyboard in the laptop. Sometimes, certain keystrokes are not being processed
  by the keyboard. We have tested this on the laptops using the xev tool, and we
  cannot figure out the reason to why it sometimes fails. Certain media keys,
  such as volume controls, do not work either.
\\ \\
  I request you to look into these issues and release an update which resolves
  them by the end of this week. If this is not possible, I wish to know your
  course of action in resolving this issue and a date at which you expect it will
  be resolved. Failure to do so will require me to send back the entire batch of
  laptops and demand a refund.

  \paragraph{}
  Yours faitfully,
  \\ \\
  Patrick Spek \\
  Chief Technical Officer, \\
  Scriptkitties
\end{document}


	\newpage
	\subsection{Final revision}
	\documentclass[11pt,english]{article}

\usepackage[english]{babel}

\begin{document}
  \textbf{Subject: Laptop issues}

  \paragraph{}
  Dear Sir or Madam,
\\ \\
  I am writing to inform you about issues we have with the laptops we recently
  ordered from your company, Lenovo.
\\ \\ 
  I am afraid that the batch of 600 IdeaPads we recently ordered and received
  are having technical issues which are impairing our current workflow. The
  laptops seem to have issues connecting to our WiFi network, even after we
  applied all firmware updates.
\\ \\
  Another issue we have ran into is battery failure. Even though ACPI reports
  the batteries to have well over one hour of battery life left, the machine
  shuts off and will only turn on when the laptop is connected to a power source
  or a fully charged battery is inserted.
\\ \\
  I have also received a number of complaints from my employees about the
  keyboard in the laptop. Sometimes, certain keystrokes are not being processed
  by the keyboard. We have tested this on the laptops using the xev tool, and we
  cannot figure out the reason to why it sometimes fails. Certain media keys,
  such as volume controls, do not work either.
\\ \\
  I request you to look into these issues and release an update which resolves
  them by the end of this week. If this is not possible, I wish to know your
  course of action in resolving this issue and a date at which you expect it will
  be resolved. Failure to do so will require me to send back the entire batch of
  laptops and demand a refund.

  \paragraph{}
  Yours faitfully,
  \\ \\
  Patrick Spek \\
  Chief Technical Officer, \\
  Scriptkitties
\end{document}


	\newpage
	\section{The executive summary}
	\subsection{Old revision}
	\documentclass[11pt,english]{article}

\usepackage[english]{babel}

\begin{document}
Timaflu is a small company in the medicine sector. They have a large building in
order to grow, and they wish to start growing at this point in time. There are
some issues that prevent them from expanding. VTA2 is a group of students to
solve their problems so they can expand as per Timaflu's vision.
\\ \\
Currently, Timaflu is using paper documents for all internal communication and
processing orders from their customers. The customers place their orders using
phonecalls with Timaflu employees.
\\ \\
VTA2 has found that their current methods of using paper documents is their
biggest hurdle in expanding the company. Therefore it is advised to implement a
computer information system to handle these documents instead. This will allow
them to avoid a lot of the errors that may arise from using paper documents.
Adding to this, such a system can easily generate reports for the board of
directors at Timaflu.
\\ \\
As such, VTA2 will start on designing a database structure for this project. The
system will have to deal with all paper documents which are in use right now. It
must also be able to provide an overview of all processes and orders going on
and generate reports of the business.
\end{document}


	\newpage
	\subsection{Final revision}
	\documentclass[11pt,english]{article}

\usepackage[english]{babel}

\begin{document}
Timaflu is a small company in the medicine sector. They have a large building in
order to grow, and they wish to start growing at this point in time. There are
some issues that prevent them from expanding. VTA2 is a group of students to
solve their problems so they can expand as per Timaflu's vision.
\\ \\
Currently, Timaflu is using paper documents for all internal communication and
processing orders from their customers. The customers place their orders using
phone calls with Timaflu employees.
\\ \\
VTA2 has found that their current methods of using paper documents is their
biggest hurdle in expanding the company. Therefore it is advised to implement a
computer information system to handle these documents instead. This will allow
them to avoid a lot of the errors that may arise from using paper documents.
Adding to this, such a system can easily generate reports for the board of
directors at Timaflu.
\\ \\
As such, VTA2 will start on designing a database structure for this project. The
system will have to deal with all paper documents which are in use right now. It
must also be able to provide an overview of all processes and orders going on
and generate reports of the business.
\end{document}


	\newpage
	\section{The technical tutorial}
	\subsection{Old revision}
	\documentclass[11pt,english]{article}

\usepackage[english]{babel}
\usepackage{enumitem}
\usepackage{fancyhdr}
\usepackage[top=2.3cm, bottom=2.3cm, left=1.6cm, right=1.6cm]{geometry}
\usepackage{lastpage}
\usepackage{standalone}

\setlength{\parindent}{0pt}        % indentation on new paragraph
\setlength{\parskip}{0pt}          % vertical spacing on new paragraph
\setlength{\lineskip}{1pt}         % vertical spacing between lines
\setlength{\columnsep}{1cm}        % spacing between columns
\setlength{\belowcaptionskip}{0pt} % spacing below captions
\setlength{\abovecaptionskip}{5pt} % spacong above captions

\pagestyle{fancy}

\lfoot{Patrick Spek}
\cfoot{\today}
\rfoot{\thepage/\pageref{LastPage}}

\newcommand*{\Frontpage}{\begingroup
	\hbox{%
		\hspace*{0.2\textwidth}
		\rule{1pt}{\textheight}
		\hspace*{0.05\textwidth}
		\parbox[b]{0.75\textwidth}{%
			{\noindent\Huge\bfseries Portfolio}\\[2\baselineskip] % Title
			{\large \textit{English writing}}\\[4\baselineskip] % Tagline or further description
			{\Large \textsc{\\
				Patrick Spek \\
			}}

			\vspace{0.5\textheight} % Whitespace between the title block and the publisher
		}
	}
\endgroup}

\renewcommand{\footrulewidth}{0.4pt}

% fixes for the markdown stuff
\usepackage{hyperref}
\providecommand{\tightlist}{\setlength{\itemsep}{0pt}\setlength{\parskip}{0pt}}

\begin{document}
	\thispagestyle{empty}
	\Frontpage{}
	\newpage

	\tableofcontents
	\newpage

	$body$
\end{document}



	\newpage
	\subsection{Final revision}
	\begin{document}

\section{Installing Funtoo}\label{installing-funtoo}

It has come to my attention that many people consider installing Gentoo,
and in effect, Funtoo, a hard task to complete. Some people have also
shown interest in my particular setup.

As such, I have written this tutorial to show people my installation
steps. If you have any suggestions or criticism, please find me on IRC.
The networks I frequent and the nickname I use can be found on
\href{http://tyil.work}{my homepage}.

\subsection{Assumptions}\label{assumptions}

This tutorial assumes a few things from you. If you do not meet most of
these assumptions, this guide is probably not for you. You can of course
still read it, however, there might be much jargon you do not
understand, making the tutorial more complex to understand.

\begin{itemize}
\tightlist
\item
  You have experience with GNU+Linux
\item
  You know your way in the terminal
\item
  You are not afraid of using text-based applications
\item
  You have experience reading through manuals and documentation
\item
  You are not afraid to spend some hours on IRC to help you troubleshoot
  issues
\end{itemize}

\subsection{Installing Funtoo}\label{installing-funtoo-1}

This tutorial will guide you through a not-so-basic installation of the
Funtoo GNU+Linux distribution. It is based on one of my own
installations, but slightly simplified.

\subsubsection{The live environment}\label{the-live-environment}

Before you can get started with setting up the system, you will need
something to set it up with. We will be using a live environment for
this purpose. My personal choice for this task is
\href{http://www.system-rescue-cd.org/SystemRescueCd_Homepage}{the
Gentoo-based SystemRescueCD}.

You can use any other live environment your prefer, however, this
tutorial will only guide you into preparing the System Rescue CD.

\paragraph{Getting the live USB image}\label{getting-the-live-usb-image}

You can download the System Rescue CD at one of the following locations:

\begin{itemize}
\tightlist
\item
  \href{http://build.funtoo.org/distfiles/sysresccd/systemrescuecd-x86-4.7.1.iso}{Funtoo}
\item
  \href{http://ftp.osuosl.org/pub/funtoo/distfiles/sysresccd/systemrescuecd-x86-4.7.1.iso}{Osuosl}
\end{itemize}

\paragraph{Setting up the live USB}\label{setting-up-the-live-usb}

After downloading the image, mount it somewhere:

\begin{verbatim}
mount path/to/sysrescuecd.iso /mnt/cdrom
\end{verbatim}

Once it is mounted, you can run the installer bundled with the image by
running

\begin{verbatim}
/mnt/cdrom/usb_inst.sh
\end{verbatim}

Select the right device and wait for the installer to finish up.

\paragraph{Booting the USB}\label{booting-the-usb}

To begin using the live environment so you can install something with
it, boot it up. Make sure the USB is in the machine, and reboot it.
Enter the BIOS/UEFI settings and make sure to either make the USB device
a higher boot priority, or select it to be the boot device for one boot.
The availability and location of these options differs per machine, so
be sure to check the manual or look around online for instructions if it
is not clear to you.

\subsubsection{Hardware preparation}\label{hardware-preparation}

The hardware you are installing on needs to be prepared. This could mean
manually configuring your hardware RAID if you use this and configuring
other exotic setups. This tutorial will not go into details for such
setups, as there is a near infinite amount of possible options. Instead,
you should stick to simply configuring your storage device.

The size of your storage device should be at least 35GB to be safe and
have some space for personal data. The partitioning layout this guide is
aiming for is the following:

\begin{verbatim}
DEVICE                 FILE SYSTEM  SIZE  MOUNT POINT
sda
  sda1                 fat32        2GB  /boot
  sda2                 lvm
    funtoo0-root       xfs          8GB  /
    funtoo0-home       zfs               /home
    funtoo0-sources    ext4         3GB  /usr/src
    funtoo0-portage    reiserfs     2GB  /usr/portage
    funtoo0-swap       swap
    funtoo0-packages   xfs         10GB  /var/packages
    funtoo0-distfiles  xfs         10GB  /var/distfiles
\end{verbatim}

If you already an advanced user, you are of course free to diverge from
the guide here.

\paragraph{Partition the drive}\label{partition-the-drive}

The first part is to setup partitions. This can be done by calling

\texttt{gdisk\ /dev/sda}

Let us wipe the entire disk and start with a clean slate. You can do
this by typing \texttt{o} and pressing enter. When asked whether you are
sure, type \texttt{y} and enter again.

Now you are going to create two partitions, one for \texttt{/boot} and
one for
\href{https://en.wikipedia.org/wiki/Logical_Volume_Manager_(Linux)}{\texttt{lvm}}.
Following is a list of what to enter.
\texttt{\textless{}CR\textgreater{}} denotes pressing the enter key.

\begin{itemize}
\item
  \texttt{n} \texttt{\textless{}CR\textgreater{}}
\item
  \texttt{\textless{}CR\textgreater{}}
\item
  \texttt{\textless{}CR\textgreater{}}
\item
  \texttt{+500M} \texttt{\textless{}CR\textgreater{}}
\item
  \texttt{EF00} \texttt{\textless{}CR\textgreater{}}
\item
  \texttt{n} \texttt{\textless{}CR\textgreater{}}
\item
  \texttt{\textless{}CR\textgreater{}}
\item
  \texttt{\textless{}CR\textgreater{}}
\item
  \texttt{\textless{}CR\textgreater{}}
\item
  \texttt{\textless{}CR\textgreater{}}
\end{itemize}

\paragraph{Setting up encryption}\label{setting-up-encryption}

Any system should be safe. Encryption is just a small part, but in my
opinion definitely important. We are going to encrypt the entire
\texttt{lvm} partition using
\href{https://en.wikipedia.org/wiki/Linux_Unified_Key_Setup}{\texttt{luks}}.
The front end tool to be used for this is
\href{https://en.wikipedia.org/wiki/Dm-crypt\#cryptsetup}{\texttt{cryptsetup}}:

\begin{verbatim}
cryptsetup --cipher aes-xts-plain64 --hash sha512 --key-size 256 luksFormat /dev/sda2
\end{verbatim}

\texttt{cryptsetup} will ask you for a pass phrase. Make sure to use a
good one, preferably at least 20 characters in length.

Once the partition has been encrypted, open the device so it can be used
by invoking \texttt{cryptsetup\ luksOpen\ /dev/sda2\ dmcrypt\_lvm}.

\paragraph{Set up LVM}\label{set-up-lvm}

Once the encrypted partition has been unlocked, you can setup
\texttt{lvm} on it. To initialize an lvm volume on this partition, run
the following:

\begin{verbatim}
pvcreate /dev/mapper/dmcrypt_lvm
vgcreate funtoo0 /dev/mapper/dmcrypt_lvm
\end{verbatim}

The lvm volume has now been prepared, and you can start adding volumes
to it to be used as partitions. It is recommended to have a swap
partition as well. The size of this partition depends on the amount of
RAM you have available. Due to my availability to big disks, I generally
opt for a swap partition the same size as my total RAM in the machine.
To make the tutorial work for this as well, a sub shell is called to
figure out the size of the swap partition.

\begin{verbatim}
lvcreate -L8G -n root funtoo0
lvcreate -L3G -n sources funtoo0
lvcreate -L2G -n portage funtoo0
lvcreate -L10G -n packages funtoo0
lvcreate -L10G -n distfiles funtoo0
lvcreate -L$(free | grep -i mem: | awk '{print $2}') -n swap funtoo0
lvcreate -l 100%FREE -n home funtoo0
\end{verbatim}

\paragraph{Create file systems}\label{create-file-systems}

Now you are ready to create usable file systems on the partitions:

\begin{verbatim}
mkfs.vfat -F32 /dev/sda1
mkfs.xfs /dev/mapper/funtoo0-root
mkfs.xfs /dev/mapper/funtoo0-packages
mkfs.xfs /dev/mapper/funtoo0-distfiles
mkfs.reiserfs /dev/mapper/funtoo0-portage
mkfs.ext4 /dev/mapper/funtoo0-sources
mkswap /dev/mapper/funtoo0-swap
\end{verbatim}

If you're thinking at this point ``where's my home partition?'', it's
not initialized here. \href{https://en.wikipedia.org/wiki/ZFS}{ZFS}
requires custom kernel modules which will be built later, after the
initial kernel has been compiled.

\paragraph{Mount the file systems}\label{mount-the-file-systems}

Next up is mounting all file systems so you can install files to them.
First, you mount the root file system:

\begin{verbatim}
mount /dev/mapper/funtoo0-root /mnt/gentoo
\end{verbatim}

Now you can add some directories for the other mount points. This can be
done in one well-made \texttt{mkdir} invocation:

\begin{verbatim}
mkdir -p /mnt/gentoo/{boot,home,usr/{portage,src},var/{tmp,distfiles,packages},tmp}
\end{verbatim}

Next you can mount all other mount points on the new directories:

\begin{verbatim}
mount /dev/sda1 /mnt/gentoo/boot
mount /dev/mapper/funtoo0-portage /mnt/gentoo/usr/portage
mount /dev/mapper/funtoo0-sources /mnt/gentoo/usr/src
mount /dev/mapper/funtoo0-distfiles /mnt/gentoo/var/distfiles
mount /dev/mapper/funtoo0-packages /mnt/gentoo/var/packages
\end{verbatim}

Let's also enable swap and ram disks for the temporary storage
directories:

\begin{verbatim}
swapon /dev/mapper/funtoo0-swap
mount -t tmpfs none /mnt/gentoo/tmp
mount --rbind /mnt/gentoo/tmp /mnt/gentoo/var/tmp
\end{verbatim}

\subsubsection{Initial setup}\label{initial-setup}

Now that all mount points have been set up, installation of the actual
OS can begin. This is done by downloading a ``stage 3'' tarball
containing a bare minimal Funtoo installation and extracting it with the
right options.

The stage 3 tarball can be downloaded from
\href{http://build.funtoo.org/}{build.funtoo.org}. It is easiest to
download and extract the tarball in the root file system, so let's do
that:

\begin{verbatim}
cd /mnt/gentoo
wget http://build.funtoo.org/funtoo-current/x86-64bit/generic_64/stage3-latest.tar.xz
tar xpf stage3-latest.tar.xz
\end{verbatim}

Once extraction is complete, you can opt to delete the tarball as it is
no longer needed at this point. You can delete it by invoking
\texttt{rm\ stage3-latest.tar.gz}.

\subsubsection{System configuration}\label{system-configuration}

You now have a bare Funtoo installation ready on your machine. But
before you can actually use it, you should do some configuration.

\paragraph{Chrooting}\label{chrooting}

Before you get to the configuration part, you should
\href{https://en.wikipedia.org/wiki/Chroot}{\texttt{chroot}} into the
system. This allows you to enter your new Funtoo installation before it
can properly boot. If your system ever breaks and you are unable to boot
into it anymore, you can redo the mounting section of this guide and
this chrooting section to get into it and resolve your issues.

The chrooting requires a couple extra mounts, so the chroot can
interface with the hardware provided by the system above it:

\begin{verbatim}
mount -t proc none proc
mount --rbind /dev dev
mount --rbind /sys sys
\end{verbatim}

Once these mount points are set, you will need to copy over
\texttt{resolv.conf} so the chroot can resolve DNS names:

\begin{verbatim}
cp /etc/resolv.conf etc
\end{verbatim}

Now that everything is prepared in the chroot, you can enter your Funtoo
installation using the following:

\begin{verbatim}
chroot . bash -l
\end{verbatim}

\paragraph{Set up the portage tree}\label{set-up-the-portage-tree}

The portage tree is a collection of files which are used by the package
manager to find out which software it can install, and more importantly,
how to install it.

The default location in Funtoo for your portage tree is in
\texttt{/usr/portage}. However, as I use multiple sources for my portage
tree, I prefer to set it up under \texttt{/usr/portage/funtoo}. This is
not a required step, but advised nonetheless.

In order to change this, open up \texttt{/etc/portage/repos.conf/gentoo}
in your favorite editor. Funtoo comes with
\href{https://en.wikipedia.org/wiki/Vi}{\texttt{vi}},
\href{https://en.wikipedia.org/wiki/GNU_nano}{\texttt{nano}} and
\href{https://en.wikipedia.org/wiki/Ed_(text_editor)}{\texttt{ed}} by
default. \texttt{ed} is recommended as the standard editor. After
opening the file, change the \texttt{location} key to point to
\texttt{/usr/portage/funtoo}.

When you have modified \texttt{/etc/portage/repos.conf/gentoo} (or not,
if you do not want to change this default), continue to download your
first version of the portage tree:

\begin{verbatim}
emerge --sync
\end{verbatim}

Every time you want to update your system, you will have to do an
\texttt{emerge\ -\/-sync} to update the portage tree first. It is
managed by \href{https://en.wikipedia.org/wiki/Git}{\texttt{git}}, which
can bring some side effects. The most notable one is that the tree will
grow over time with old commit data. If you wish to clean this up,
simply \texttt{rm\ -rf\ /usr/portage/*\ \&\&\ emerge\ -\/-sync} to
regenerate it from scratch

\paragraph{Setting up your system
settings}\label{setting-up-your-system-settings}

In order to make the system work properly, some setup has to be
performed. This will involve editing some text files, for which you can
use your favorite editor again.

\subparagraph{/etc/fstab}\label{etcfstab}

We will begin with the most important one, \texttt{/etc/fstab}. This
file holds information on your mount points. Some of the mount points
are best configured with UUIDs, because the device enumeration can
sometimes differ. If you have multiple storage devices in your system,
this could as well be a hard requirement. UUIDs are unique to each
storage device, so you will have to figure out your UUIDs yourself. You
can do this by running \texttt{lsblk\ -o\ +UUID}. Take note of the UUID
of your boot device.

Once you know the UUID, open up \texttt{/etc/fstab} with whatever editor
you feel comfortable with and make it look like the following block of
text. Do not forget to update the UUIDs!

\begin{verbatim}
# boot device
/dev/sda1  /boot  vfat  noauto,noatime  1 2

# lvm volumes
/dev/mapper/funtoo0-root       /               xfs       rw,relatime,data=ordered  0 1
/dev/mapper/funtoo0-portage    /usr/portage    reiserfs  defaults                  0 0
/dev/mapper/funtoo0-sources    /usr/src        ext4      noatime                   0 1
/dev/mapper/funtoo0-packages   /var/packages   xfs       defaults                  0 1
/dev/mapper/funtoo0-distfiles  /var/distfiles  xfs       defaults                  0 1

# ramdisks
tmpfs  /tmp  tmpfs  defaults  0 0

# swap
/dev/mapper/funtoo0-swap  none  swap  defaults  0 0

# binds
/tmp  /var/tmp  none  rbind  0 0
\end{verbatim}

\subparagraph{/etc/localtime}\label{etclocaltime}

The localtime comes next. This is to make sure your time is set
correctly. An incorrect time can cause issues such as secure connections
failing. To set your localtime, all you need to do is create a symlink.
The file you need to symlink to is stored in
\texttt{/usr/share/zoneinfo}. The files are sorted by continent. As
someone who lives in the Netherlands, I'd use
\texttt{/usr/share/zoneinfo/Europe/Amsterdam}:

\begin{verbatim}
ln -fs /usr/share/zoneinfo/Europe/Amsterdam /etc/localtime
\end{verbatim}

It is important to also correctly set your hardware clock, in case it is
off. Check if your time and date are correct by invoking \texttt{date}.
If these settings are correct, you can skip towards the next heading.
Otherwise, keep on reading this bit.

To set the correct time, you can use the \texttt{date} utility again.
When invoked with an argument in the form of \texttt{MMDDhhmmYYYY}, it
will set the date and time instead of check it. The following command
would set the date to the first of October 2016, and the time to 17:29:

\begin{verbatim}
date 011017292016
\end{verbatim}

After you correctly set the date and time to whatever it currently is,
sync it to the hardware clock so it is correct across reboots:

\begin{verbatim}
hwclock --systohc
\end{verbatim}

\subparagraph{/etc/portage/make.conf}\label{etcportagemake.conf}

Another important part to configure is the \texttt{make.conf} file. This
file contains settings for portage and some options for compilers. This
file can also be made a directory. This way, you can split off your
configurations into multiple files for easier maintenance. The files
will be loaded alphabetically. The way you set it up is completely up to
you, though I would recommend removing the default
\texttt{/etc/portage/make.conf} and making it a directory instead.

Once you have decided how to setup your make.conf, it is time to add
some data in the file(s). Following is a list of useful variables to set
up, with a block containing my own settings for it. You can copy these
for yourself, or dig around some man pages to find out what you exactly
want yourself.

USE

\texttt{USE} holds global USE flags. These are used to configure your
packages. You can turn features on and off using these, and the ebuilds
will configure the packages to enable or disable these features.

\begin{verbatim}
USE="
    ${USE}
    alsa
    gtk
    gtkstyle
    infinality
    vim-syntax
    zsh-completion
    -pulseaudio
    -systemd
    -gnome
    -kde
"
\end{verbatim}

FEATURES

The \texttt{FEATURES} variable allows enabling of various portage
features. Mine are setup to drop privileges so root is used as little as
possible and to do as much parallel as possible to speed up the process.
Additionally, I use the \texttt{buildpkg} feature to build binary
packages for use on other systems. This can save you a great deal of
time if you have multiple systems running Funtoo.

\begin{verbatim}
FEATURES="
    ${FEATURES}
    buildpkg
    network-sandbox
    parallel-fetch
    parallel-install
    sandbox
    userfetch
    userpriv
    usersandbox
    usersync
"
\end{verbatim}

EMERGE\_DEFAULT\_OPTS

\texttt{EMERGE\_DEFAULT\_OPTS} can be used to add some flags to every
emerge you invoke. This way you can force emerge to always ask for
confirmation.

\begin{verbatim}
EMERGE_DEFAULT_OPTS="
    ${EMERGE_DEFAULT_OPTS}
    --alert
    --ask
    --binpkg-changed-deps=y
    --binpkg-respect-use=y
    --keep-going
    --tree
    --usepkg
    --verbose
"
\end{verbatim}

C/XXFLAGS

The \texttt{CFLAGS} and \texttt{CXXFLAGS} variables hold
compiler-specific options. It is \textbf{incredibly} important to not
use newlines in these two, as
\href{https://bugs.gentoo.org/show_bug.cgi?id=500034\#c6}{they will
break \texttt{cmake}}. Other than that, it is just a regular shell
variable like the others.

\begin{verbatim}
CFLAGS="-O2 -pipe"
CXXFLAGS="-O2 -pipe"
\end{verbatim}

ACCEPT\_LICENSE

This variable is not as important as the others. You can even opt to
leave it out completely. If, however, you wish to limit portage to only
install free software (free as in freedom, not gratis), you can set it
to the same value as me. Do note that if you use this, you will need to
setup the \texttt{/etc/portage/package.license} as well.

\begin{verbatim}
ACCEPT_LICENSE="
    -*
    @FREE
"
\end{verbatim}

MAKEOPTS

\texttt{MAKEOPTS} are the arguments passed to \texttt{make}. This can be
used to instruct \texttt{make} to use multiple threads when compiling
software. The amount of threads can be set with the \texttt{-j} flag.
The general rule of thumb for this is to use
\texttt{\$((\$(nproc)\ +\ 1))}.

\begin{verbatim}
MAKEOPTS="
    -j9
"
\end{verbatim}

PKG/DISTDIR

The \texttt{PKGDIR} and \texttt{DISTDIR} variables set the location to
store binary packages after building, and the location to store
distfiles. In order to use the \texttt{/var/distfiles} and
\texttt{/var/packages} partitions, these must be set.

\begin{verbatim}
DISTDIR=/var/distfiles
PKGDIR=/var/packages
\end{verbatim}

\subparagraph{/etc/portage/package.mask}\label{etcportagepackage.mask}

Like the \texttt{make.conf} file, \texttt{package.mask} can be made a
directory containing separate files.

The \texttt{package.mask} file(s) allow you to ``mask'' packages,
instructing portage to ignore these. It can also let you mask certain
versions of packages. This way you can skip a broken version or stick to
a certain version for whatever reason. Since this tutorial uses ZFS,
there is such a reason to do exactly that.

ZFS requires a Linux at version 4.4 or lower. The latest kernel is much
higher than that, so it is necessary to mask newer kernel versions. This
is a single line of configuration, and as such can be done without a
fancy editor. Simply invoke the following magic:

\begin{verbatim}
mkdir -p /etc/portage/package.mask
echo ">sys-kernel/*-sources-4.4.6" > /etc/portage/package.mask/20-zfs.mask
\end{verbatim}

\subparagraph{/etc/portage/package.license}\label{etcportagepackage.license}

This file can be setup as a directory too, just like \texttt{make.conf}
and \texttt{package.mask}. Using this file or directory you can add
per-package license exceptions. This is therefore only needed if you
setup a strict license limit. The kernel comes with some sources under
the \texttt{freedist} license, which is not part of \texttt{@FREE}. As
such, if you want to install kernel sources you will have to make an
exception for this license on this package.

\begin{verbatim}
mkdir -p /etc/portage/package.license
echo "sys-kernel/* freedist" # TODO: check if this works
\end{verbatim}

\subparagraph{/etc/conf.d/hostname}\label{etcconf.dhostname}

As one of the last files to setup, the hostname should be set in
\texttt{/etc/conf.d/hostname}. The \texttt{hostname} variable in this
file should be set to the hostname of the machine. You can pick any name
you like, but should be unique across your network.

\paragraph{Preparing your first
kernel}\label{preparing-your-first-kernel}

Every system needs a kernel, a piece of software to interface with the
hardware. Funtoo, like every GNU+Linux distribution, uses the Linux
kernel for this task.

For this task, you will first need to decide on a source set to use. All
source sets share the same base, but they have different patches
applied. It is recommended to use \texttt{sys-kernel/gentoo-sources}. If
this isn't bleeding edge enough, you can use
\texttt{sys-kernel/git-sources} instead. If you just want the latest
official kernel without the gentoo patch set, pick
\texttt{sys-kernel/vanilla-sources}. No matter which source set you use,
the compilation and installation process remains the same.

Install whichever source set you want to use, this guide will use
\texttt{sys-kernel/gentoo-sources}. In order to save some yes-pressing
later on, the \texttt{emerge} command here will install some additional
packages which are needed for the system to function properly.

\begin{verbatim}
emerge boot-update cryptsetup lvm2 gentoo-sources
genkernel --menuconfig --lvm --luks all
\end{verbatim}

The \texttt{genkernel} command will run the kernel menuconfig utility.
If you have exotic hardware that needs special support, this is the
place to enable it. The defaults are sane for most systems. If you have
nothing to configure here, just exit the menuconfig and let
\texttt{genkernel} build a custom kernel and initramfs for you. As the
guide uses LVM and LUKS, you will need to have support for these things
in your kernel. You will need to enable the following options at the
least:

\begin{verbatim}
General setup --->
    [*] Initial RAM filesystem and RAM disk (initramfs/initrd) support
\end{verbatim}

\begin{verbatim}
Device Drivers --->
    Generic Driver Options --->
        [*] Maintain a devtmpfs filesystem to mount at /dev
\end{verbatim}

\begin{verbatim}
Device Drivers --->
    [*] Multiple devices driver support --->
        <*>Device Mapper Support
        <*> Crypt target support
\end{verbatim}

\begin{verbatim}
Cryptographic API --->
    <*> XTS support
    -*-AES cipher algorithms
\end{verbatim}

\paragraph{Setup ZFS}\label{setup-zfs}

The kernel is now installed at \texttt{/boot}, and all the required
parts to build custom kernel modules are available. This means it is now
possible to build the ZFS modules.

First install the kernel module, then format the partition, and
configure ZFS to work with it.

\begin{verbatim}
emerge zfs
modprobe zfs # TODO: check if this part and further actually works in the chroot
zpool create funtooz /dev/sda
zfs create -o mountpoint=/home funtooz/home
\end{verbatim}

\paragraph{Installing a boot loader}\label{installing-a-boot-loader}

Before building your kernel, \texttt{boot-update} was installed. This
pulls in \href{https://en.wikipedia.org/wiki/GNU_GRUB}{\texttt{grub}},
the recommended boot loader for Funtoo. It doesn't require much
configuration thanks to the \texttt{boot-update} script, which will
configure \texttt{grub} for you.

Before running the script, there's one place to update as this setup
uses \texttt{luks} and \texttt{lvm}.

Open up \texttt{/etc/boot.conf} in your favorite editor and let the file
display something like this:

\begin{verbatim}
boot {
    generate grub
    default "Funtoo GNU+Linux"
    timeout 3
}

"Funtoo GNU+Linux" {
    kernel kernel[-v]
    initrd initramfs[-v]
    params += crypt_root=/dev/sda2 real_root=/dev/mapper/funtoo0-root rootfstype=xfs dolvm
}
\end{verbatim}

Now that \texttt{boot-update} is configured, install \texttt{grub} as an
UEFI boot loader and generate the configurations for it using
\texttt{boot-update}.

\begin{verbatim}
grub-install --target=x86_64-efi --efi-directory=/boot --bootloader-id="Funtoo GNU+Linux" --recheck /dev/sda
boot-update
\end{verbatim}

\paragraph{Set your system profile}\label{set-your-system-profile}

Your system is now ready to boot and use. However, some things are still
not configured. These can in some cases be configured after rebooting,
but it is recommended to fix it all up now. The first part is setting
your system profile.

For a full list of settings, check \texttt{epro\ list}. Maybe you want
to use this system as something other than a workstation, or want to
enable the \texttt{gnome} mix-in.

To get the same profile settings as I use for my work environments, run
the following:

\begin{verbatim}
epro flavor workstation
epro mix-ins +no-systemd
\end{verbatim}

\paragraph{Running the first full system
update}\label{running-the-first-full-system-update}

The stage 3 tarball may have been the latest, but it might still have
some slightly outdated packages. In addition, now that your system
profile is set up, some applications may be configured to have different
feature sets enabled. To make sure everything is in the best possible
state, it is recommended to run a full system update now. Since some of
our options are already set as \texttt{EMERGE\_DEFAULT\_OPTS}, this is
as simple as

\begin{verbatim}
emerge -uDN @world
\end{verbatim}

\paragraph{Installing supporting
software}\label{installing-supporting-software}

This is software you will more than likely need on any standard system.
If you're an advanced user you can decide to skip this and make your own
choices, otherwise it is recommended to install this software as well.

\begin{verbatim}
emerge connman sudo vim linux-firmware
\end{verbatim}

\paragraph{Configuring supporting
software}\label{configuring-supporting-software}

Some of the supporting software has to be turned on explicitly or have a
configuration file tweaked. If you opted to not use a given recommended
package, you can skip the section with the same name.

\subparagraph{connman}\label{connman}

\href{https://en.wikipedia.org/wiki/ConnMan}{\texttt{connman}} is a
simple \textbf{conn}ection \textbf{man}ager. It's lightweight, fast and
does its job pretty well. To enable this service at boot, run

\begin{verbatim}
rc-update add connman default
\end{verbatim}

If you want to setup wireless connection authentication credentials,
read up on \texttt{man\ connman-service.conf}.

\subparagraph{sudo}\label{sudo}

The \href{https://en.wikipedia.org/wiki/Sudo}{\texttt{sudo}} utility
allows certain users, based on their username or groups they belong to,
access to privileged commands. It can also be used to run a command as a
different user. The most basic setup allows people from the
\texttt{wheel} group to execute commands normally reserved for
\texttt{root}.

Because sudo is a critical utility, it comes with its own editor that
basically just wraps your preferred editor in a script that will
complain if the configuration is wrong. To use this tool, invoke

\begin{verbatim}
visudo
\end{verbatim}

Scroll to the line which contains \texttt{\#\ \%wheel\ ALL=(ALL)\ ALL},
and remove the \texttt{\#}.

\paragraph{Create a user}\label{create-a-user}

Create a user for yourself on the system. You can use any other value
for \texttt{tyil} if you so desire:

\begin{verbatim}
useradd -m -g users -G wheel tyil
\end{verbatim}

The \texttt{-G\ wheel} part is optional, but recommended if you wish to
use this account for administrative tasks. This option adds the user to
the \texttt{wheel} group, which will allow the user to execute root
commands using \texttt{sudo}.

\paragraph{Set passwords}\label{set-passwords}

We probably want to be able to log in to the system as well. By default,
users without passwords are disabled, so you'll need to set a password
for the users you want to be able to use:

\begin{verbatim}
passwd root
passwd tyil
\end{verbatim}

If you used a different user name than \texttt{tyil}, be sure to change
it here as well.

\subsubsection{First boot}\label{first-boot}

Installation is now finished, so it is time to boot into your new Funtoo
system. First you should cleanly unmount all partitions and then issue a
reboot:

\begin{verbatim}
exit
cd
umount -lR /mnt/gentoo
reboot
\end{verbatim}

If you set your UEFI to favor the USB system over the standard drive in
the booting order, be sure to either change this back, or simply remove
the USB device.

\subsection{What's next}\label{whats-next}

Now you have a working Funtoo installation. Next steps would be
installing all the software you wish to use and configuring it to your
liking. I would greatly advise looking at other people's configurations
and publishing your configurations as well. These configuration
collections are often called \emph{dotfiles}. Mine can be found
\href{https://c.darenet.org/tyil/dotfiles-gohan}{on c.darenet.org}.

If this is your first time using Funtoo as your distro of choice, I
would recommend looking through
\href{http://www.funtoo.org/Funtoo_Linux_First_Steps}{Funtoo (GNU+)Linux
First Steps} on the official Funtoo wiki.

If you need assistance on maintenance, you can always drop by in
\texttt{\#sqt} on \href{https://freenode.net}{Gratisnode}.

\end{document}


	\newpage
	\section{Reflection}
	The writing exercises were all pretty straightforward, yet had quite a few
	catches to improve on. My main issue was with the deadlines being
	irregular, which led to me not having feedback on the first two
	assignments. I did attend the classes, and thus did get general information
	on the subjects.

	\paragraph{}
	My main issue was with the executive summary, as this was required to be on
	a document from the previous year, which I did no longer have. I did
	receive an unfinished version from a student I did the assignment with, but
	this led to a very small summary.

	\subsection{Going forward}
	To improve further on my English skills, I intend to keep on writing
	documents in English and have it reviewed by native speakers. I have since
	published the tutorial exercise on my own website, and have plans to write
	more articles.

	Due to the technical matter of these articles, this will increase my
	English skills especially for work related discussions and presentations.
	This is in line with the required B2 level of my university, and is
	therefore exactly the kind of practice I need.
\end{document}

