\documentclass[11pt,english]{article}

\usepackage[english]{babel}
\usepackage{enumitem}
\usepackage{fancyhdr}
\usepackage{graphicx}
\usepackage{lastpage}

\pagestyle{fancy}

\graphicspath{{assets/}}

\lfoot{Patrick Spek}
\cfoot{\today}
\rfoot{\thepage/\pageref{LastPage}}

\newcommand*{\Frontpage}{\begingroup
	\hbox{%
		\hspace*{0.2\textwidth}
		\rule{1pt}{\textheight}
		\hspace*{0.05\textwidth}
		\parbox[b]{0.75\textwidth}{%
			{\noindent\Huge\bfseries Gesprekken}\\[2\baselineskip] % Title
			{\large \textit{Week 3}}\\[4\baselineskip] % Tagline or further description
			{\Large \textsc{\\
				Patrick Spek \\
			}}

			\vspace{0.5\textheight} % Whitespace between the title block and the publisher
		}
	}
\endgroup}

\renewcommand{\footrulewidth}{0.4pt}

\begin{document}
	% frontpage
	\thispagestyle{empty}
	\Frontpage{}

	% table of contents
	\newpage
	\tableofcontents

	% main
	\newpage
	\section{Conflictgesprek 1}
	\subsection{Doel}
	Raoul rustig krijgen, het volledige verhaal uitleggen en zorgen dat Ruud en
	Raoul weer goed met elkaar overweg kunnen.

	\subsection{Strategie}
	Raoul aanspreken, aangeven dat ze heeft gehoord dat hij boos is op Ruud,
	Raoul zijn verhaal laten doen, daarna Ruud's kant van het verhaal aan Raoul
	vertellen.

	\subsection{Onderdelen}
	Zelfde als strategie.

	\newpage
	\section{Conflictgesprek 2}
	\subsection{Rol en taken}
	Als medewerker heb je te maken met je chef. Zijn taak is om de afdeling
	goed te laten lopen. Je hebt ook te maken met je medewerkers. Zij hebben
	ieder hun eigen rol, en zonder jou hebben zij meer werk te doen.

	\subsection{Belang van de chef}
	De chef heeft als belang om de afdeling goed te laten lopen.

	\subsection{Doel}
	Vrij krijgen op de dag van de bruiloft.

	\subsection{Strategie}
	Het gesprek voorzichtig aansnijden zodat het geen gevoelige snaar raakt bij
	de chef. Daarna kijken of er valt te onderhandelen: avonden later
	doorwerken om op de betreffende dag zelf vrij te krijgen.
\end{document}

