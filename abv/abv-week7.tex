\documentclass[11pt,english]{article}

\usepackage[english]{babel}
\usepackage{enumitem}
\usepackage{fancyhdr}
\usepackage[top=3.3cm, bottom=3.3cm, left=2.6cm, right=2.6cm]{geometry}
\usepackage{graphicx}
\usepackage{lastpage}
\usepackage{tabularx}

\pagestyle{fancy}
\setlength{\parindent}{0pt}        % indentation on new paragraph
\setlength{\parskip}{0pt}          % vertical spacing on new paragraph
\setlength{\lineskip}{1pt}         % vertical spacing between lines
\setlength{\columnsep}{1cm}        % spacing between columns
\setlength{\belowcaptionskip}{0pt} % spacing below captions
\setlength{\abovecaptionskip}{5pt} % spacong above captions

\graphicspath{{assets/}}

\lfoot{}
\cfoot{\today}
\rfoot{\thepage/\pageref{LastPage}}

\newcommand*{\Frontpage}{\begingroup
	\hbox{%
		\hspace*{0.2\textwidth}
		\rule{1pt}{\textheight}
		\hspace*{0.05\textwidth}
		\parbox[b]{0.75\textwidth}{%
			{\noindent\Huge\bfseries ABV}\\[2\baselineskip] % Title
			{\large \textit{Huiswerkopdrachten, week 7}}\\[4\baselineskip] % Tagline or further description
			{\Large \textsc{\\
				Patrick Spek, 2099745 \\
			}}
			\vspace{0.5\textheight} % Whitespace between the title block and the publisher
		}
	}
\endgroup}

\renewcommand{\footrulewidth}{0.4pt}

\begin{document}
	\thispagestyle{empty}
	\Frontpage

	\newpage
	\tableofcontents

	\newpage
	\section{Huidige Tuckman fase}
	De huidige fase waarin de groep zich bevind is naar mijn mening nog wat
	lastig om te zeggen. Er zijn nog wel wat conflicten, wat zou duiden op de
	\textbf{storming} fase. Aan de andere kant zijn er ook momenten waarop
	\textbf{norming} en soms zelfs \textbf{performing} als fase zou kunnen
	worden gezien.

	Voor storming was er het voorval waarbij een aantal gele kaarten werden
	uitgedeeld wat resulteerde in een conflict, maar daarna was er met een
	aantal leden norming te zien; er werden wat afspraken verduidelijkt en men
	werkte door aan zijn of haar opdrachten. Bij samenkomsten hierna was er tussen
	groepsleden ook duidelijk performing te zien

	\section{De kracht van de groep}
	De kracht van onze groep zit hem in de kennis die een aantal groepsleden
	samen hebben. Hierdoor kunnen we grote inhaalslagen maken voor het project,
	wat voor ons vrij belangrijk is momenteel. Een voorbeeld hiervan is
	bijvoorbeeld Kevin die zonder al te veel moeite de algehele database kan
	updaten om conform onze standaarden te zijn, en dat er minder administratie
	vanuit de applicatie nodig is (dus minder werk voor de rest van de groep).

	\section{De valkuil van de groep}
	Echter hebben we ook valkuilen. De meest duidelijke is waarschijnlijk de
	conflictsituaties die moeilijk worden opgelost, zelfs wanneer er een docent
	bij is. Dit zorgt voor onrust in de groep en kost ons tijd die we aan het
	ontwikkelen hadden kunnen besteden. Als voorbeeld hiervoor is de uitreiking
	van de laatste reeks gele kaarten in de groep.

	\section{De belangrijkste afspraak}
	De belangrijkste afspraak die wij nu naleven is het op tijd zijn. Dit is
	onder andere op tijd zijn voor samenkomsten, maar ook het opleveren van de
	opdrachten die gemaakt moesten worden. In samenwerking met Ger wordt hier
	ook streng op gelet, zodat iedereen zijn of haar werk correct en op tijd
	uitvoert. Een andere belangrijke afspraak die we recent hebben vastgelegd is
	het tonen van eigen initiatief.
\end{document}

