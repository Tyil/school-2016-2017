\documentclass[11pt,english]{article}

\usepackage[english]{babel}
\usepackage{enumitem}
\usepackage{fancyhdr}
\usepackage{graphicx}
\usepackage{lastpage}

\pagestyle{fancy}

\graphicspath{{assets/}}

\lfoot{}
\cfoot{\today}
\rfoot{\thepage/\pageref{LastPage}}

\newcommand*{\Frontpage}{\begingroup
  \hbox{%
    \hspace*{0.2\textwidth}
    \rule{1pt}{\textheight}
    \hspace*{0.05\textwidth}
    \parbox[b]{0.75\textwidth}{%
      {\noindent\Huge\bfseries IDPRI}\\[2\baselineskip] % Title
      {\large \textit{Huiswerkopdrachten}}\\[4\baselineskip] % Tagline or further description
      {\Large \textsc{\\
          Patrick Spek, 2099745 \\
          Roy, 2097591 \\
        }}
      \vspace{0.5\textheight} % Whitespace between the title block and the publisher
    }
  }
\endgroup}

\renewcommand{\footrulewidth}{0.4pt}

\begin{document}
  \thispagestyle{empty}
  \Frontpage
  \newpage

  \tableofcontents
  \newpage

  \section{Opdracht 1}
  \begin{enumerate}
    \item De gebruiker kan nieuwe bugs toevoegen.
    \item De gebruiker kan naar bugs zoeken in het systeem.
      \begin{itemize}
        \item Name
        \item ID
        \item Reporter
        \item Assignee
        \item Priority
        \item Age
      \end{itemize}
    \item De gebruiker kan bugs filteren in het systeem.
      \begin{itemize}
        \item Name
        \item ID
        \item Reporter
        \item Assignee
        \item Priority
        \item Age
      \end{itemize}
    \item De gebruiker kan bugs sorteren.
      \begin{itemize}
        \item Name
        \item ID
        \item Reporter
        \item Assignee
        \item Priority
        \item Age
      \end{itemize}
    \item De gebruiker kan een rapportage met de voortgang van het project
      opvragen.
    \item De gebruiker kan versies van de applicatie met bijbehorende bugs
      beheren.
    \item Het systeem kan onderscheid maken tussen verschillende gebruikers en
      hierop gebaseerd bepaalde rechten geven.
    \item De gebruiker kan de status van bugs aanpassen.
    \item De gebruiker wil de historie van bugs in kunnen zien.
    \item De gebruiker kan producten beheren.
  \end{enumerate}

  \newpage
  \section{Opdracht 2}
  \subsection{Navigatiestructuur}
  \makebox[\textwidth]{\includegraphics[width=\linewidth]{navigatiemodel}}

  \subsection{Navigatiemodel}
  We maken gebruik van het multi-level navigatiemodel.

  \subsection{Andere navigatiemodellen}
  Andere navigatiemodellen die we toepassen zijn:
  \begin{itemize}
    \item \textbf{Escape hatch} om direct terug te keren naar de homepage.
    \item \textbf{Bookmarks} om naar een specifieke bug of project te kunnen
      refereren, bijvoorbeeld in de descripion van een andere bug.
  \end{itemize}
\end{document}

