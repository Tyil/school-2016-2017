\documentclass[11pt,english]{article}

\usepackage[english]{babel}
\usepackage{enumitem}
\usepackage{fancyhdr}
\usepackage{graphicx}
\usepackage{lastpage}

\pagestyle{fancy}

\graphicspath{{assets/}}

\lfoot{Patrick Spek}
\cfoot{\today}
\rfoot{\thepage/\pageref{LastPage}}

\newcommand*{\Frontpage}{\begingroup
	\hbox{%
		\hspace*{0.2\textwidth}
		\rule{1pt}{\textheight}
		\hspace*{0.05\textwidth}
		\parbox[b]{0.75\textwidth}{%
			{\noindent\Huge\bfseries Gesprekken}\\[2\baselineskip] % Title
			{\large \textit{Week 1}}\\[4\baselineskip] % Tagline or further description
			{\Large \textsc{\\
				Patrick Spek \\
			}}

			\vspace{0.5\textheight} % Whitespace between the title block and the publisher
		}
	}
\endgroup}

\renewcommand{\footrulewidth}{0.4pt}

\begin{document}
	% frontpage
	\thispagestyle{empty}
	\Frontpage{}

	% table of contents
	\newpage
	\tableofcontents

	% main
	\newpage
	\section{Luistertest}
	\subsection{Resultaten}
	\begin{tabular}{ l l }
		Mensgericht & 0 \\
		Handelingsgericht & 0 \\
		Inhoudsgericht & 1 \\
		Tijdgericht & 1 \\
	\end{tabular}

	\subsection{Herkenning}
	Gezien de lage hoeveelheid aan 4 en 5 waardering bij stellingen kan ik hier
	niet echt iets uit herkennen. Volgens de waardering zou ik zowel
	inhoudsgericht en tijdgericht moeten zijn. Dit kan ik wel redelijk in
	mijzelf herkennen, maar het handelingsgerichte deel zie ik ook in mijzelf
	terug.

	Als voorbeeld heb ik werksituaties. Wanneer er een nieuwe opdracht is
	probeer ik bij de opdrachtgever zoveel mogelijk informatie over de nodige
	handelingen te verkrijgen, in een zo kort mogelijke tijd. Dit is zowel
	handelingsgericht als tijdgericht.

	\newpage
	\section{Samenvatting}
	Lichaamstaal is belangrijk. Het kan veel verraden over de huidige mentale
	toestand van een persoon, en over de waarheid van wat ze vertellen. Deze
	kun je aflezen aan zogeheten ``microexpressions''. Volgens recent onderzoek
	is er gemiddeld maar 7\% van een gesprek verteld in woorden. De rest van de
	informatie haal je uit lichaamstaal en intonatie. Zeker wanneer iemand
	onder druk staat kan lichaamstaal veel weggeven over wat een persoon nu
	daadwerkelijk zegt.

	Een voorbeeld van lichaamstaal is de manier waarop iemand beweegt. Iemand
	die stevig doorloopt geeft hiermee een krachtige status af. Ook is de
	volgorde waarin een groep mensen door een deur heen loopt van belang. Dit
	is wel iets moeilijker om een verband aan te leggen gezien dit per cultuur
	en situatie verschilt.

	Ook hoe mensen elkaar de hand schudden is van belang, zeker wanneer er
	fotosessies zijn. In dat geval wil de persoon die zichzelf als
	belangrijker, of sterker, wil neerzetten altijd links staan op de foto.
	Hierdoor is de hand van deze persoon boven op de hand van de andere persoon
	te zien op foto\'s.

	De manier en volgorde waarop een groep personen zit is ook belangrijk. De
	persoon die in het midden van de groep zit trekt altijd de meeste aandacht,
	en lijkt hierdoor het belangrijkst. Wanneer iemand breed gaat zitten, maar
	ook iets op schoot vast houdt met zijn handen, is dit een teken dat deze
	persoon sterk probeert over te komen, maar van zichzelf al weet dat hij dat
	niet is.

	Een belangrijk concept om lichaamstaal te kunnen gebruiken bij het
	beoordelen van een persoon is ``norming''. Dit houdt in dat je gaat
	onderzoeken wat de normale houding is van specifieke personen zodat je een
	norm kunt stellen. Wanneer iemand afwijkt van zijn normale houding valt dit
	dan makkelijker op. Afwijkend gedrag kan dan aanduiden dat er iets mis is
	dat de persoon probeert te verbergen.
\end{document}

