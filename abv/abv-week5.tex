\documentclass[11pt,english]{article}

\usepackage[english]{babel}
\usepackage{enumitem}
\usepackage{fancyhdr}
\usepackage[top=3.3cm, bottom=3.3cm, left=2.6cm, right=2.6cm]{geometry}
\usepackage{graphicx}
\usepackage{lastpage}
\usepackage{tabularx}

\pagestyle{fancy}
\setlength{\parindent}{0pt}        % indentation on new paragraph
\setlength{\parskip}{0pt}          % vertical spacing on new paragraph
\setlength{\lineskip}{1pt}         % vertical spacing between lines
\setlength{\columnsep}{1cm}        % spacing between columns
\setlength{\belowcaptionskip}{0pt} % spacing below captions
\setlength{\abovecaptionskip}{5pt} % spacong above captions

\graphicspath{{assets/}}

\lfoot{}
\cfoot{\today}
\rfoot{\thepage/\pageref{LastPage}}

\newcommand*{\Frontpage}{\begingroup
	\hbox{%
		\hspace*{0.2\textwidth}
		\rule{1pt}{\textheight}
		\hspace*{0.05\textwidth}
		\parbox[b]{0.75\textwidth}{%
			{\noindent\Huge\bfseries ABV}\\[2\baselineskip] % Title
			{\large \textit{Huiswerkopdrachten, week 5}}\\[4\baselineskip] % Tagline or further description
			{\Large \textsc{\\
				Patrick Spek, 2099745 \\
			}}
			\vspace{0.5\textheight} % Whitespace between the title block and the publisher
		}
	}
\endgroup}

\renewcommand{\footrulewidth}{0.4pt}

\begin{document}
	\thispagestyle{empty}
	\Frontpage

	\newpage
	\tableofcontents

	\newpage
	\section{De Bono kleuren van de groep}
	\begin{itemize}
		\item \textbf{Jonathan} Blauw
		\item \textbf{Kevin} Geel
		\item \textbf{Kris} Geel
		\item \textbf{Merel} Geel
		\item \textbf{Mohammed} Groen
	\end{itemize}

	\subsection{Voorbeelden}
	\subsubsection{Jonathan}
	Van Jonathan is het duidelijk dat hij het proces belangrijk vind zodat er
	een product wordt opgeleverd waar een voldoende voor te krijgen is. In een
	gesprek met Ger heeft hij letterlijk gezegd dat dat zijn enige interesse is
	tijdens het project.

	\subsubsection{Mohammed}
	Mohammed is redelijk vrij in in zijn manier van doen en denken. Dit leidt
	soms tot conflicten, zoals vaker te laat zijn dan de rest. Hij wil
	daarentegen ook ieders mening horen zodat hij een goede positie kan innemen
	bij een argument. Hiervoor is het een vereiste om open te zijn en open te
	staan voor andere ideeen.

	\subsubsection{Kevin}
	Bij Kevin merk je duidelijk dat hij altijd de positieve onderdelen van
	alles probeert in te zien. Elk probleem kent een oplossing en hij probeert
	zo snel mogelijk problemen te verhelpen zodat de groep weer op positieve
	voet verder kan gaan.

	\newpage
	\section{De Bono kleur van mijzelf}
	De De Bono kleur die ik mijzelf toeken is \textbf{wit}. Ik weet van mijzelf
	dat ik objectieve informatie wil hebben om beslissingen te maken,
	emotionele waarde is zinloos in programmacode. Ik zou mezelf ook groen
	kunnen toekennen gezien ik altijd open sta voor alternatieven, mits deze
	goed worden onderbeuwd. Gezien ik deze onderbouwing weer volledig objectief
	wil hebben (dus wit), ben ik van mening dat wit beter bij mij past dan
	groen.

	\subsection{Voorbeeld}
	Een goed voorbeeld is de vorige ABV meeting, waarbij ik door middel van
	objectieve informatie tot de conclusie kwam dat een van onze groepsleden
	een idioot was. Hoewel dit objectief gezien een correcte conclusie is,
	mocht dit helaas niet gemeld worden. Het leidt dus helaas wel tot
	conflicten.
\end{document}

