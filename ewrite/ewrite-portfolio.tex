\documentclass[11pt,english]{article}

\usepackage[english]{babel}
\usepackage{enumitem}
\usepackage{fancyhdr}
\usepackage[top=2.3cm, bottom=2.3cm, left=1.6cm, right=1.6cm]{geometry}
\usepackage{graphicx}
\usepackage{lastpage}
\usepackage{standalone}

\setlength{\parindent}{0pt}        % indentation on new paragraph
\setlength{\parskip}{0pt}          % vertical spacing on new paragraph
\setlength{\lineskip}{1pt}         % vertical spacing between lines
\setlength{\columnsep}{1cm}        % spacing between columns
\setlength{\belowcaptionskip}{0pt} % spacing below captions
\setlength{\abovecaptionskip}{5pt} % spacong above captions

\pagestyle{fancy}

\graphicspath{{assets/}}

\lfoot{Patrick Spek}
\cfoot{\today}
\rfoot{\thepage/\pageref{LastPage}}

\newcommand*{\Frontpage}{\begingroup
	\hbox{%
		\hspace*{0.2\textwidth}
		\rule{1pt}{\textheight}
		\hspace*{0.05\textwidth}
		\parbox[b]{0.75\textwidth}{%
			{\noindent\Huge\bfseries Portfolio}\\[2\baselineskip] % Title
			{\large \textit{English writing}}\\[4\baselineskip] % Tagline or further description
			{\Large \textsc{\\
				Patrick Spek \\
			}}

			\vspace{0.5\textheight} % Whitespace between the title block and the publisher
		}
	}
\endgroup}

\renewcommand{\footrulewidth}{0.4pt}

% additional markup for included documents
\usepackage{hyperref}
\providecommand{\tightlist}{\setlength{\itemsep}{0pt}\setlength{\parskip}{0pt}}

\begin{document}
	% frontpage
	\thispagestyle{empty}
	\Frontpage{}

	% foreword
	\newpage
	\section{Foreword}
	\par
	This document is the portfolio for my English writing course at Avans. It
	is intended to help me learn about writing in a professional manner which I
	can use in my later career. At the start of our course I was asked to write
	down a number of goals I wanted to improve on during this course.

	\par
	One of these goals is to improve my spelling. While there are dictionaries
	available to correct my mistakes, it would save me time and effort to
	simply do it right in a single try. This is something I wish to improve on
	during the course.

	\par
	CONTENT

	\par
	I tend to write a lengthy piece of text once I get started on it. However,
	this can be a chore to read for the other party, and even make the intended
	message harder to read. In turn, I want to improve on this by making
	concise messages that clearly get the intended message across, while still
	leaving enough details for the recipient to understand the reasoning behind
	the message.

	\par
	Another of my learning goals for this class is to make less errors with
	regards to the word ordering. In formal English texts (and not-so-formal
	texts as well, for that matter), the order in which words appear is
	important. I often make mistakes in this area, for instance by putting the
	time somewhere in the middle of the sentence instead of at the end.

	\par
	In the rest of this document you will find my results of the assignments I
	was given for this course. This includes three different documents. The
	first is an email with a complaint. The second assignment is an executive
	summary on a document made for an assignment in the first year of
	university. Lastly, there is a tutorial document. This is a rather tough
	piece of text intended for those with a little more experience in the IT
	field.

	\par
	In addition to these finalized assignments, this portfolio also contains
	their earlier versions. These were handed in to be rated by our instructor
	so that I can improve on it.

	% table of contents
	\newpage
	\tableofcontents

	% main body
	\newpage
	\section{The Email}
	\subsection{Old revision}
	\documentclass[11pt,english]{article}

\usepackage[english]{babel}

\begin{document}
  \textbf{Subject: Laptop issues}

  \paragraph{}
  Dear Sir or Madam,
\\ \\
  I am writing to inform you about issues we have with the laptops we recently
  ordered from your company, Lenovo.
\\ \\ 
  I am afraid that the batch of 600 IdeaPads we recently ordered and received
  are having technical issues which are impairing our current workflow. The
  laptops seem to have issues connecting to our WiFi network, even after we
  applied all firmware updates.
\\ \\
  Another issue we have ran into is battery failure. Even though ACPI reports
  the batteries to have well over one hour of battery life left, the machine
  shuts off and will only turn on when the laptop is connected to a power source
  or a fully charged battery is inserted.
\\ \\
  I have also received a number of complaints from my employees about the
  keyboard in the laptop. Sometimes, certain keystrokes are not being processed
  by the keyboard. We have tested this on the laptops using the xev tool, and we
  cannot figure out the reason to why it sometimes fails. Certain media keys,
  such as volume controls, do not work either.
\\ \\
  I request you to look into these issues and release an update which resolves
  them by the end of this week. If this is not possible, I wish to know your
  course of action in resolving this issue and a date at which you expect it will
  be resolved. Failure to do so will require me to send back the entire batch of
  laptops and demand a refund.

  \paragraph{}
  Yours faitfully,
  \\ \\
  Patrick Spek \\
  Chief Technical Officer, \\
  Scriptkitties
\end{document}


	\newpage
	\subsection{Final revision}

	\newpage
	\subsection{Old revision}
	\section{The executive summary}
	\documentclass[11pt,english]{article}

\usepackage[english]{babel}

\begin{document}
Timaflu is a small company in the medicine sector. They have a large building in
order to grow, and they wish to start growing at this point in time. There are
some issues that prevent them from expanding. VTA2 is a group of students to
solve their problems so they can expand as per Timaflu's vision.
\\ \\
Currently, Timaflu is using paper documents for all internal communication and
processing orders from their customers. The customers place their orders using
phonecalls with Timaflu employees.
\\ \\
VTA2 has found that their current methods of using paper documents is their
biggest hurdle in expanding the company. Therefore it is advised to implement a
computer information system to handle these documents instead. This will allow
them to avoid a lot of the errors that may arise from using paper documents.
Adding to this, such a system can easily generate reports for the board of
directors at Timaflu.
\\ \\
As such, VTA2 will start on designing a database structure for this project. The
system will have to deal with all paper documents which are in use right now. It
must also be able to provide an overview of all processes and orders going on
and generate reports of the business.
\end{document}


	\newpage
	\subsection{Final revision}

	\newpage
	\subsection{Old revision}
	\section{The technical tutorial}
	\documentclass[11pt,english]{article}

\usepackage[english]{babel}
\usepackage{enumitem}
\usepackage{fancyhdr}
\usepackage[top=2.3cm, bottom=2.3cm, left=1.6cm, right=1.6cm]{geometry}
\usepackage{lastpage}
\usepackage{standalone}

\setlength{\parindent}{0pt}        % indentation on new paragraph
\setlength{\parskip}{0pt}          % vertical spacing on new paragraph
\setlength{\lineskip}{1pt}         % vertical spacing between lines
\setlength{\columnsep}{1cm}        % spacing between columns
\setlength{\belowcaptionskip}{0pt} % spacing below captions
\setlength{\abovecaptionskip}{5pt} % spacong above captions

\pagestyle{fancy}

\lfoot{Patrick Spek}
\cfoot{\today}
\rfoot{\thepage/\pageref{LastPage}}

\newcommand*{\Frontpage}{\begingroup
	\hbox{%
		\hspace*{0.2\textwidth}
		\rule{1pt}{\textheight}
		\hspace*{0.05\textwidth}
		\parbox[b]{0.75\textwidth}{%
			{\noindent\Huge\bfseries Portfolio}\\[2\baselineskip] % Title
			{\large \textit{English writing}}\\[4\baselineskip] % Tagline or further description
			{\Large \textsc{\\
				Patrick Spek \\
			}}

			\vspace{0.5\textheight} % Whitespace between the title block and the publisher
		}
	}
\endgroup}

\renewcommand{\footrulewidth}{0.4pt}

% fixes for the markdown stuff
\usepackage{hyperref}
\providecommand{\tightlist}{\setlength{\itemsep}{0pt}\setlength{\parskip}{0pt}}

\begin{document}
	\thispagestyle{empty}
	\Frontpage{}
	\newpage

	\tableofcontents
	\newpage

	$body$
\end{document}



	\newpage
	\subsection{Final revision}

	\newpage
	\section{Reflection}
\end{document}

